\chapter {Distribuci�n como juego online}

Las aventuras de AGE se pueden jugar online, directamente desde el
navegador, a trav�s de un applet que viene con el propio AGE.

Para distribuir una aventura de AGE de esta manera, es requisito
indispensable que los accesos a ficheros que haga la aventura (im�genes,
m�sica, etc.) se hagan mediante el m�todo \textsf{getResource()} de la
clase \textsf{World}.

Cumplido este requisito, distribuir una aventura de AGE para jugar online
es muy f�cil. S�lo hay que hacer lo siguiente:

\begin{enumerate}
\item {Copiar en un directorio el fichero \textsf{AgeCore.jar} y el
directorio \textsf{lib del} AGE.}
\item {Con el directorio \textsf{worlds} que contenga la aventura se pueden
tomar dos alternativas diferentes: o bien copiarlo tambi�n junto al
\textsf{AgeCore.jar} y al directorio \textsf{lib}; o bien meterlo dentro del
fichero \textsf{AgeCore.jar}. Si se quiere usar esta segunda alternativa,
se hace as�:
	\begin{enumerate}
	\item {Abrir el fichero \textsf{AgeCore.jar} con un programa de manejo de
	ficheros zip (el formato \textsf{.jar} es lo mismo que el formato
	\textsf{.zip}, pero renombrado. Si el programa descompresor de ficheros
	zip no quiere abrirlo, puedes cambiarle la extensi�n temporalmente a zip).}
	\item {Meter el directorio \textsf{worlds} que contenga la aventura dentro
	del \textsf{AgeCore.jar}.}
	\end{enumerate}}
\item {Crear un fichero .html basado en el ejemplo que sigue:
\lstset{language=html}
\begin{lstlisting}
<html>
<body>
<center></center>
</body>
</html>
 
<html> 
<body> 
<div align="center" style='position: relative; min-height: 95%; min-width: 95%'> 
<applet code = "eu.irreality.age.swing.applet.SwingSDIApplet"
    archive = "AgeCore.jar,lib/bsh-2.0b4.jar,lib/commons-cli-1.2.jar", 
    width = "750", 
    height = "95%"
    align = "center">
    <param name="worldUrl" value="worlds/Morluck/world.xml"/>
    <param name="java_arguments" value="-Xmx300M"/>
</applet>
</div> 
</body> 
</html>
\end{lstlisting}

Haciendo los cambios siguientes:}
\item {Donde pone \comillas{worlds/Morluck/world.xml}, debes cambiarlo a la ruta
correspondiente al fichero \textsf{world.xml} de tu aventura dentro del
archivo \textsf{.jar}.}
\item {Donde pone \comillas{AgeCore.jar,lib/bsh-2.0b4.jar,lib/commons-cli-1.2.jar},
debes a�adir (separadas por comas) las rutas a las bibliotecas adicionales
que utilice tu aventura. Por ejemplo, si tu aventura tiene m�sica en formato
ogg, tendr�s que a�adir la biblioteca que toca m�sica en formato ogg. He
aqu� una tabla de las bibliotecas que necesitas a�adir en cada caso:

\begin{tabular}{|l|l|}
\hline
\textbf{biblioteca} & \textbf{cu�ndo hace falta} \\
\hline
\hline
\textsf{lib/bsh-2.0b4.jar} & Necesaria siempre \\
\hline
\textsf{lib/commons-cli-1.2.jar} & Necesaria siempre \\
\hline
\textsf{lib/basicplayer3.0.jar} & Aventuras con audio mp3, ogg y/o speex \\
\hline
\textsf{lib/commons-logging-api.jar} & Aventuras con audio mp3, ogg y/o speex \\
\hline
\textsf{lib/jl1.0.jar} & Aventuras con audio mp3, ogg y/o speex \\
\hline
\textsf{lib/tritonus\_share.jar} & Aventuras con audio mp3, ogg y/o speex \\
\hline
\textsf{lib/jogg-0.0.7.jar} & Aventuras con audio ogg \\
\hline
\textsf{lib/jorbis-0.0.15.jar} & Aventuras con audio ogg \\
\hline
\textsf{lib/vorbisspi1.0.2.jar} & Aventuras con audio ogg \\
\hline
\textsf{lib/jspeex0.9.7.jar} & Aventuras con audio speex \\
\hline
\textsf{lib/micromod.jar} & Aventuras con audio en formato mod \\
\hline
\textsf{lib/mp3spi1.9.4.jar} & Aventuras con audio en formato mp3 \\
\hline
\textsf{lib/svgSalamander.jar} & Aventuras con gr�ficos vectoriales svg \\
\hline
\end{tabular}
}
\end{enumerate}

Una vez completados estos pasos, deber�as poder jugar la aventura
localmente, abriendo en tu navegador la p�gina HTML que has creado. S�lo
falta el directorio completo (incluyendo la p�gina web, el fichero
\textsf{AgeCore.jar} y el subdirectorio \textsf{lib}) a tu servidor web, y
ya se podr� jugar a tu aventura a trav�s de internet, accediendo a la URL
de la p�gina HTML.

N�tese que si quieres crear p�ginas para jugar varios mundos distintos, no
necesitas subir al servidor distintos directorios con copias del fichero
\textsf{AGECore.jar} y el subirectorio \textsf{lib}. Basta con subir una
sola copia, que todos los mundos a jugar est�n en el directorio
\textsf{worlds} (est� �ste metido dentro de \textsf{AgeCore.jar} o fuera),
y crear una p�gina HTML para cada mundo que especifique la ruta al mismo en
el par�metro \textsf{worldUrl}.

La diferencia entre poner el directorio worlds dentro del
\textsf{AgeCore.jar} o fuera est� en los tiempos de carga. Si se pone
dentro, el fichero \textsf{.jar}, que es lo que hay que descargarse al
empezar a jugar la aventura, ser� m�s grande. Esto supone una espera inicial
m�s larga; pero a cambio, los posibles elementos multimedia que tenga el
juego se mostrar�n m�s r�pido una vez superada esta espera inicial, ya que
est�n todos dentro del \textsf{.jar}. Si se pone fuera, suceder� lo
contrario: la espera inicial ser� m�s corta; pero a cambio, cada vez que se
utilice un nuevo fichero multimedia habr� que bajarlo del servidor por lo
que se producir� una espera adicional.
