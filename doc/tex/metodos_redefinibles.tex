\chapter {M�todos redefinibles}

\section {Eventos}

\section {Otros m�todos redefinibles}

\subsection {M�todos de Mobile (para PSIs)} \label{sec:metodos_psis}

\begin {itemize}
\item {M�todos de combate
	\begin {itemize}
	\item {\verb|void addEnemy ( Mobile m )|: a�adirle a un enemigo, al que
	atacar� autom�ticamente en cuanto lo vea.}
	\item {\verb|void attack ( Mobile m , Weapon w )|: atacar a otra criatura
	con el arma dada.}
	\item {\verb|void block ( Mobile m , Weapon w )|: bloquear.}
	\item {\verb|void cast ( Spell s , Entity e )|: conjurar el hechizo dado
	sobre la entidad dada.}
	\item {\verb|void die()|: morir.}
	\item {\verb|void dodge ( Mobile m )|: esquivar.}
	\end {itemize}
}
\item {M�todos de movimiento
	\begin {itemize}
	\item {\verb|boolean goTo ( Room r )|: ir a la habitaci�n dada (contigua
	a la actual).}
	\item {\verb|boolean makeRandomValidMove()|: moverse aleatoriamente hacia
	alguna habitaci�n accesible desde la actual.}
	\end {itemize}
}
\item {M�todos para hablar y reaccionar
	\begin {itemize}
	\item {\verb|void say ( String s )|: decir el texto dado.}
	\item {\verb|void say ( String s , String style )|: decir el texto dado,
	y que los mensajes de notificaci�n correspondientes se impriman en el
	estilo dado.}
	\item {\verb|void onSay ( Mobile aCreature , String text )|: qui�n ha
	hablado y que ha dicho. ej: decir \comillas{hola}}
	\item {\verb|void onSayTo ( Mobile speaker , String text , Mobile hearer )|:
	qui�n ha hablado, que ha dicho y a quien. ej: decir \comillas{hola} a Gandalf.}
	\item {\verb|void onRoomText ( String text )|: para que el PSI pueda
	reaccionar al texto que se env�a a la habitaci�n.}
	\end {itemize}
}
\end {itemize}

\subsection {Ejemplo de uso}

Estos ejemplos (si no se dice lo contrario) van en la pesta�a
\textit {C�digo y propiedades} del PSI.

Ejemplo de detecci�n de texto enviado a la habitaci�n. Cogemos una pelota,
y el PSI se enfada, puesto que es suya.

\begin{lstlisting}
void onRoomText ( String text )
{
  if ( text.contains("coge la pelota") )
  {
      self.say("�Eh! �Esa pelota es m�a!\n�Ya no me ajunto contigo!\n");
  }
}
\end{lstlisting}

