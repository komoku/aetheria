\chapter {Referencia de m�todos invocables}

\section {Manipulaci�n del modelo de mundo}

\section {Notificaci�n de acciones y sucesos}

\subsection {Notificar sobre algo que ha ocurrido en una habitaci�n}

En la clase \textsf{Room} tenemos los siguientes m�todos:

\begin{lstlisting}
void reportAction ( Entity source /*$1*/ , Entity target /*$2*/ , Entity[] objects /*$3..$n*/ , String thirdPersonDes , String sufferDes , String execDes , boolean self_included )
void reportAction ( Entity source /*$1*/ , Entity target /*$2*/ , Entity[] objects /*$3..$n*/ , String thirdPersonDes , String sufferDes , String execDes , String style , boolean self_included )
\end{lstlisting}

Informa a todos los jugadores presentes en la habitaci�n de un suceso que
se ha producido, donde:

\begin{itemize}
\item {\textsf{source} es el sujeto del suceso (por ejemplo, Juan),}
\item {\textsf{target} es el objeto del suceso (por ejemplo, un goblin),}
\item {\textsf{objects} son otras entidades que hayan intervenido en el
suceso (por ejemplo, una espada),}
\item {\textsf{thirdPersonDes} es la descripci�n en tercera persona con los
objetos parametrizados con \textsf{\$1} (sujeto), \textsf{\$2} (objeto),
\textsf{\$3}...\textsf{\$n} (resto). Por ejemplo: ``\$1 ataca a \$2 con
\$3''.}
\item {\textsf{sufferDes} es la descripci�n que se mostrar� al objeto del
suceso (``\$1 te ataca con \$3'').}
\item {\textsf{execDes} es la descripci�n que es mostrar� al sujeto del
suceso (``Atacas a \$1 con \$3'').}
\item {\textsf{style} permite especificar un estilo (color) para mostrar el
texto a los jugadores (puede ser ``story'', ``description'', ``action'',
etc.)}
\item {\textsf{self\_included}: si es \textsf{true}, se muestra el texto
tambi�n para el sujeto, si no, s�lo para el resto de jugadores/criaturas.}
\end{itemize}

\begin{lstlisting}
void reportActionAuto ( Entity source /*$1*/ , Entity target /*$2*/ , Entity[] objects /*$3..$n*/ , String thirdPersonDes , boolean self_included )
void reportActionAuto ( Entity source /*$1*/ , Entity target /*$2*/ , Entity[] objects /*$3..$n*/ , String thirdPersonDes , String style , boolean self_included )
\end{lstlisting}

Hace lo mismo que \textsf{reportAction}; pero s�lo proporcionamos la
descripci�n en tercera persona y AGE intenta generar las otras dos mediante
sus tablas de ``tercera a segunda''. Utilizar s�lo en los dos casos
siguientes:

\begin{itemize}
\item {Si realmente no es necesario el grado de detalle que da
\textsf{reportAction}, pues no hay que emitir descripciones distintas en
primera, segunda y tercera persona (por ejemplo, en juegos monojugador).}
\item {Si s� es necesario emitir descripciones distintas en primera,
segunda y tercera persona; pero la oraci�n es lo suficientemente sencilla
como para que AGE pueda convertirla (en el futuro se dar�n m�s detalles de
cu�les se pueden convertir autom�ticamente y cu�les no, de momento se puede
saber por prueba y error).}
\end{itemize}

\subsection {Notificar sobre algo que ha ocurrido con una cosa}

En la clase Item tenemos los siguientes m�todos:

\begin{lstlisting}
void reportAction ( Entity source /*$1*/ , Entity target /*$2*/ , Entity[] objects /*$3..$n*/ , String thirdPersonDes , String sufferDes , String execDes  , boolean self_included )
void reportAction ( Entity source /*$1*/ , Entity target /*$2*/ , Entity[] objects /*$3..$n*/ , String thirdPersonDes , String sufferDes , String execDes , String style , boolean self_included )
void reportActionAuto ( Entity source /*$1*/ , Entity target /*$2*/ , Entity[] objects /*$3..$n*/ , String thirdPersonDes ,  boolean self_included )
void reportActionAuto ( Entity source /*$1*/ , Entity target /*$2*/ , Entity[] objects /*$3..$n*/ , String thirdPersonDes , String style , boolean self_included )
\end{lstlisting}

Funcionan exactamente igual que los de la clase \textsf{Room}, y lo que
hacen es mostrar la notificaci�n en todas las habitaciones en las que est�
el \textsf{Item} dado.

\subsection {Notificar sobre algo que ha ocurrido con una criatura}

No hay m�todos espec�ficos en la clase \textsf{Mobile}, ya que una criatura
siempre est� en una sola habitaci�n, f�cilmente accesible mediante
\textsf{getRoom()}. Por lo tanto, podemos hacer estas notificaciones
f�cilmente as�:

\begin{lstlisting}
elMobile.getRoom().informAction(...)
elMobile.getRoom().informActionAuto(...)
\end{lstlisting}

\section {Presentaci�n general}

\section {M�todos �tiles de la API de Java}
